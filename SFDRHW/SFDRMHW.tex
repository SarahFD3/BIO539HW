\documentclass[]{article}
\usepackage{lmodern}
\usepackage{amssymb,amsmath}
\usepackage{ifxetex,ifluatex}
\usepackage{fixltx2e} % provides \textsubscript
\ifnum 0\ifxetex 1\fi\ifluatex 1\fi=0 % if pdftex
  \usepackage[T1]{fontenc}
  \usepackage[utf8]{inputenc}
\else % if luatex or xelatex
  \ifxetex
    \usepackage{mathspec}
  \else
    \usepackage{fontspec}
  \fi
  \defaultfontfeatures{Ligatures=TeX,Scale=MatchLowercase}
\fi
% use upquote if available, for straight quotes in verbatim environments
\IfFileExists{upquote.sty}{\usepackage{upquote}}{}
% use microtype if available
\IfFileExists{microtype.sty}{%
\usepackage{microtype}
\UseMicrotypeSet[protrusion]{basicmath} % disable protrusion for tt fonts
}{}
\usepackage[margin=1in]{geometry}
\usepackage{hyperref}
\hypersetup{unicode=true,
            pdftitle={SFD\_RHW},
            pdfauthor={SFD},
            pdfborder={0 0 0},
            breaklinks=true}
\urlstyle{same}  % don't use monospace font for urls
\usepackage{color}
\usepackage{fancyvrb}
\newcommand{\VerbBar}{|}
\newcommand{\VERB}{\Verb[commandchars=\\\{\}]}
\DefineVerbatimEnvironment{Highlighting}{Verbatim}{commandchars=\\\{\}}
% Add ',fontsize=\small' for more characters per line
\usepackage{framed}
\definecolor{shadecolor}{RGB}{248,248,248}
\newenvironment{Shaded}{\begin{snugshade}}{\end{snugshade}}
\newcommand{\KeywordTok}[1]{\textcolor[rgb]{0.13,0.29,0.53}{\textbf{#1}}}
\newcommand{\DataTypeTok}[1]{\textcolor[rgb]{0.13,0.29,0.53}{#1}}
\newcommand{\DecValTok}[1]{\textcolor[rgb]{0.00,0.00,0.81}{#1}}
\newcommand{\BaseNTok}[1]{\textcolor[rgb]{0.00,0.00,0.81}{#1}}
\newcommand{\FloatTok}[1]{\textcolor[rgb]{0.00,0.00,0.81}{#1}}
\newcommand{\ConstantTok}[1]{\textcolor[rgb]{0.00,0.00,0.00}{#1}}
\newcommand{\CharTok}[1]{\textcolor[rgb]{0.31,0.60,0.02}{#1}}
\newcommand{\SpecialCharTok}[1]{\textcolor[rgb]{0.00,0.00,0.00}{#1}}
\newcommand{\StringTok}[1]{\textcolor[rgb]{0.31,0.60,0.02}{#1}}
\newcommand{\VerbatimStringTok}[1]{\textcolor[rgb]{0.31,0.60,0.02}{#1}}
\newcommand{\SpecialStringTok}[1]{\textcolor[rgb]{0.31,0.60,0.02}{#1}}
\newcommand{\ImportTok}[1]{#1}
\newcommand{\CommentTok}[1]{\textcolor[rgb]{0.56,0.35,0.01}{\textit{#1}}}
\newcommand{\DocumentationTok}[1]{\textcolor[rgb]{0.56,0.35,0.01}{\textbf{\textit{#1}}}}
\newcommand{\AnnotationTok}[1]{\textcolor[rgb]{0.56,0.35,0.01}{\textbf{\textit{#1}}}}
\newcommand{\CommentVarTok}[1]{\textcolor[rgb]{0.56,0.35,0.01}{\textbf{\textit{#1}}}}
\newcommand{\OtherTok}[1]{\textcolor[rgb]{0.56,0.35,0.01}{#1}}
\newcommand{\FunctionTok}[1]{\textcolor[rgb]{0.00,0.00,0.00}{#1}}
\newcommand{\VariableTok}[1]{\textcolor[rgb]{0.00,0.00,0.00}{#1}}
\newcommand{\ControlFlowTok}[1]{\textcolor[rgb]{0.13,0.29,0.53}{\textbf{#1}}}
\newcommand{\OperatorTok}[1]{\textcolor[rgb]{0.81,0.36,0.00}{\textbf{#1}}}
\newcommand{\BuiltInTok}[1]{#1}
\newcommand{\ExtensionTok}[1]{#1}
\newcommand{\PreprocessorTok}[1]{\textcolor[rgb]{0.56,0.35,0.01}{\textit{#1}}}
\newcommand{\AttributeTok}[1]{\textcolor[rgb]{0.77,0.63,0.00}{#1}}
\newcommand{\RegionMarkerTok}[1]{#1}
\newcommand{\InformationTok}[1]{\textcolor[rgb]{0.56,0.35,0.01}{\textbf{\textit{#1}}}}
\newcommand{\WarningTok}[1]{\textcolor[rgb]{0.56,0.35,0.01}{\textbf{\textit{#1}}}}
\newcommand{\AlertTok}[1]{\textcolor[rgb]{0.94,0.16,0.16}{#1}}
\newcommand{\ErrorTok}[1]{\textcolor[rgb]{0.64,0.00,0.00}{\textbf{#1}}}
\newcommand{\NormalTok}[1]{#1}
\usepackage{graphicx,grffile}
\makeatletter
\def\maxwidth{\ifdim\Gin@nat@width>\linewidth\linewidth\else\Gin@nat@width\fi}
\def\maxheight{\ifdim\Gin@nat@height>\textheight\textheight\else\Gin@nat@height\fi}
\makeatother
% Scale images if necessary, so that they will not overflow the page
% margins by default, and it is still possible to overwrite the defaults
% using explicit options in \includegraphics[width, height, ...]{}
\setkeys{Gin}{width=\maxwidth,height=\maxheight,keepaspectratio}
\IfFileExists{parskip.sty}{%
\usepackage{parskip}
}{% else
\setlength{\parindent}{0pt}
\setlength{\parskip}{6pt plus 2pt minus 1pt}
}
\setlength{\emergencystretch}{3em}  % prevent overfull lines
\providecommand{\tightlist}{%
  \setlength{\itemsep}{0pt}\setlength{\parskip}{0pt}}
\setcounter{secnumdepth}{0}
% Redefines (sub)paragraphs to behave more like sections
\ifx\paragraph\undefined\else
\let\oldparagraph\paragraph
\renewcommand{\paragraph}[1]{\oldparagraph{#1}\mbox{}}
\fi
\ifx\subparagraph\undefined\else
\let\oldsubparagraph\subparagraph
\renewcommand{\subparagraph}[1]{\oldsubparagraph{#1}\mbox{}}
\fi

%%% Use protect on footnotes to avoid problems with footnotes in titles
\let\rmarkdownfootnote\footnote%
\def\footnote{\protect\rmarkdownfootnote}

%%% Change title format to be more compact
\usepackage{titling}

% Create subtitle command for use in maketitle
\newcommand{\subtitle}[1]{
  \posttitle{
    \begin{center}\large#1\end{center}
    }
}

\setlength{\droptitle}{-2em}

  \title{SFD\_RHW}
    \pretitle{\vspace{\droptitle}\centering\huge}
  \posttitle{\par}
    \author{SFD}
    \preauthor{\centering\large\emph}
  \postauthor{\par}
      \predate{\centering\large\emph}
  \postdate{\par}
    \date{May 10, 2019}


\begin{document}
\maketitle

\subsection{R Markdown}\label{r-markdown}

Assignment: The dataset for this assignment includes research budgets
for federal research and development agencies. The dataset comes from
the American Association for the Advancement of Science Historical
Trends. Examine the data for patterns and trends and share them in a
report such as you might give to your advisor. Your report should:
Contain plots that illustrate relationships between variables and cases
where relationships do not exist Contain plots with data divided by
category as needed Include captions and descriptions of plots Show only
informative output (not code or raw results) Format text and results
appropriately including headers and inline code. For 539 students:
Include R-squared and p-values from relevant statistical tests as
necessary to support relationships. These should be written in inline
code not output as raw results or using hardcoded numbers.

\begin{verbatim}
## -- Attaching packages ---------------------------------------------------------------------------------- tidyverse 1.2.1 --
\end{verbatim}

\begin{verbatim}
## v ggplot2 3.1.1       v purrr   0.3.0  
## v tibble  2.0.1       v dplyr   0.8.0.1
## v tidyr   0.8.3       v stringr 1.4.0  
## v readr   1.3.1       v forcats 0.4.0
\end{verbatim}

\begin{verbatim}
## Warning: package 'ggplot2' was built under R version 3.5.3
\end{verbatim}

\begin{verbatim}
## Warning: package 'stringr' was built under R version 3.5.3
\end{verbatim}

\begin{verbatim}
## -- Conflicts ------------------------------------------------------------------------------------- tidyverse_conflicts() --
## x dplyr::filter() masks stats::filter()
## x dplyr::lag()    masks stats::lag()
\end{verbatim}

\begin{verbatim}
## Warning: package 'reshape2' was built under R version 3.5.3
\end{verbatim}

\begin{verbatim}
## 
## Attaching package: 'reshape2'
\end{verbatim}

\begin{verbatim}
## The following object is masked from 'package:tidyr':
## 
##     smiths
\end{verbatim}

\begin{verbatim}
## 
## Attaching package: 'dbplyr'
\end{verbatim}

\begin{verbatim}
## The following objects are masked from 'package:dplyr':
## 
##     ident, sql
\end{verbatim}

\begin{Shaded}
\begin{Highlighting}[]
\NormalTok{fed_spend <-}\StringTok{ }\KeywordTok{read.csv}\NormalTok{(}\StringTok{"fed_r_d_spending.csv"}\NormalTok{)}
\NormalTok{energy_spend <-}\StringTok{ }\KeywordTok{read.csv}\NormalTok{(}\StringTok{"energy_spending.csv"}\NormalTok{)}
\NormalTok{climate_spend <-}\StringTok{ }\KeywordTok{read.csv}\NormalTok{(}\StringTok{"climate_spending.csv"}\NormalTok{)}

\KeywordTok{attach}\NormalTok{(fed_spend)}
\end{Highlighting}
\end{Shaded}

The first dataset contains the R\&D budget, total Federal spending,
total discretionary Federal spending, and total US Gross Domestic
Product (adjusted for inflation) for the following U.S. Government
Departments: DOD - Department of Defense, NASA - National Aeronautics
and Space Administration, DOE - Department of Energy, HHS - Department
of Health and Human Services, NIH - National Institute of Health, NSF -
National Science Foundation, USDA - US Department of Agriculture,
Interior - Department of Interior, DOT - Department of Transportation,
EPA - Environmental Protection Agency, DOC - Department of Corrections,
DHS - Department of Homeland Security, VA - Department of Veterans
Affairs, and Other - other research and development spending.

A quick plot of the discretionary outlays and total outlays revealed
that they are the same over all departments every year.

\begin{center}\includegraphics{SFDRMHW_files/figure-latex/fed_spend-1} \end{center}

\begin{center}\includegraphics{SFDRMHW_files/figure-latex/fed_spend-2} \end{center}

This is not necessarily helpful for comparision; however, when the two
values are plotted as a line (total outlays in blue and disretionary
outlays in red), it is clear how both budgets have generally increased
over time.

\includegraphics{SFDRMHW_files/figure-latex/OutlayCompare-1.pdf}

Variation is visible when the R\&D budgets are plotted.

\includegraphics{SFDRMHW_files/figure-latex/RDcompare-1.pdf}

The graph clearly indicates the the Deparment of Defense has commanded
the largest Research and Development budget over time. Even though the
budget seems to be declining in recent years, the DOD R\&D budget is
still far larger than all other reviewed departments.

Continuing our analysis of DOD spending.

\includegraphics{SFDRMHW_files/figure-latex/DODAnalysis-1.pdf}

\includegraphics{SFDRMHW_files/figure-latex/next-1.pdf}


\end{document}
